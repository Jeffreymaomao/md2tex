\subsection{標題 1}

\subsubsection{標題 2}

\subsubsubsection{標題 3}

\textbf{標題 4}

\textbf{標題 5}

\textbf{標題 6}


\begin{itemize}
	\item A 無序列表項 1
	\item B 無序列表項 2
	\begin{itemize}
		\item C 無序子列表項 1
		\item D 無序子列表項 2
	\end{itemize}
	\item E 無序列表項 1
	\item F 無序列表項 2
	\begin{enumerate}
		\item G 有序子列表項 1
		\item H 有序子列表項 2
	\end{enumerate}
	\item I 無序列表項 1
	\item J 無序列表項 2
	\begin{itemize}
		\item K 有序子列表項 1
		\begin{itemize}
			\item L 有序子列表項 2
			\item M 有序子列表項 1
		\end{itemize}
		\item N 有序子列表項 2
	\end{itemize}
\end{itemize}
\begin{enumerate}
	\item O 有序列表項 1
	\item P 有序列表項 2
	\begin{enumerate}
		\item Q 有序子列表項 1
		\item R 有序子列表項 2
	\end{enumerate}
	\item S 有序列表項 1
	\begin{enumerate}
		\item T 有序子列表項 2
	\end{enumerate}
\begin{figure}[htbp]
\includegraphics{http://example.com/image.jpg}
\caption{\textbf{image title}: 無替代文字的圖片}
\end{figure}
\end{enumerate}
\begin{equation}
y = ax+b
\end{equation}


\lstinline{行內代碼} 示例。

\vspace{5pt}
\hrule
\vspace{6pt}

\begin{equation}
y = ax+b
\end{equation}

\begin{lstlisting}% [ language = javascript ]
// 代碼塊
console.log('Hello, Markdown!');
\end{lstlisting}

\href{http://example.com}{鏈接文本}

\begin{figure}[htbp]
\includegraphics{http://example.com/image.jpg}
\caption{圖片描述}
\end{figure}

| 表格頭 | 表格頭 | 表格頭 || ------ | ------ | ------ || 單元格 | 單元格 | 單元格 || 單元格 | 單元格 | 單元格 |

\vspace{5pt}
\hrule
\vspace{6pt}

\textit{斜體文本} 或 \textit{斜體文本}

\textbf{粗體文本} 或 \textbf{粗體文本}

\textbf{\textit{粗斜體文本} 或 \textbf{\textit{粗斜體文本}

\sout{刪除線文本}

\begin{itemize}
	\item [ ] 任務列表未完成
	\item [x] 任務列表已完成
\end{itemize}
\lstinline{行內代碼}

\begin{lstlisting}
沒有指定語言的代碼塊
\end{lstlisting}

\subsection{指定語言的代碼塊}
\begin{lstlisting}% [ language = python ]
print("Hello, Markdown!")
\end{lstlisting}

$\text{如果整行都是inline-math}$

單行數學公式\begin{equation}
y = ax^2 + bx + c
\end{equation}


多行數學公式\begin{equation}
\begin{aligned}
x = \cos(t)\\
y = \sin(t)
\end{aligned}
\end{equation}



這是一個自動鏈接:\href{http://google.com}{google}

這是一張圖片 \begin{figure}[htbp]
\includegraphics{http://example.com/image.jpg}
\caption{無替代文字的圖片}
\end{figure}

數學公式 $\sqrt{x}$ 或是 $\frac{1}{2}$

普通段落文本。支持\textit{斜體}、\textbf{粗體}、\sout{刪除線}。


